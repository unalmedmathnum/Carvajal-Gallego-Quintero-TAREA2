\documentclass[12pt]{article}
\usepackage{amssymb}
\usepackage[mathscr]{euscript}
\usepackage{amsmath}
\usepackage[spanish,activeacute]{babel}
\usepackage[utf8]{inputenc}
\usepackage{enumerate}
\usepackage{graphicx}
\usepackage{listings}
\usepackage{hyperref}
\usepackage{amsmath,amsfonts,amssymb,amsthm}
\usepackage[all,cmtip]{xy}
\usepackage{tikz-cd}
\usetikzlibrary{cd}
\usetikzlibrary{arrows}
\usepackage{amsmath,amssymb,tikz-cd}
\usepackage{pstricks,pstricks-add,pst-math,pst-xkey}
%\usepackage{pstricks,pstricks-add}
\DeclareGraphicsExtensions{.bmp}
\newtheorem{definicion}{Definición}[section]
\newtheorem{teorema}{Teorema}[section]
\newtheorem{corolario}{Corolario}[section]
\newtheorem{lema}{Lema}[section]
\newtheorem{proposicion}{Proposición}[section]
\newtheorem{postulado}{Axioma}[section]
\newtheorem{observacion}{Observación}[section]
\newtheorem{ejemplo}{Ejemplo}[section]
%\DeclareTextFontComand{\Large}
\setlength{\parindent}{0pt}
\begin{document}
\begin{titlepage}
\centering
{\bfseries\LARGE Universidad Nacional de Colombia \par}
\vspace{1cm}
{\scshape\Large Facultad de Ciencias \par}
\vspace{3cm}
{\scshape\Huge M\'etdo de los M\'inimos cuadrados\par}
\vspace{3cm}
{\itshape\Large Tarea 2 
A\'alisis N\'umerico\par}
\vfill
{\Large Autor: \par}
{\Large Shara Gallego Grisales \par}
\vfill
{\Large Enero 2025 \par}
\end{titlepage}

\tableofcontents
\vspace{25.0cm}

\begin{center}
\section{Introducci\'on}
\end{center}

\vspace{0.5cm }
En este documento veremoszs y analisaremos el m\'etodo de m\'inimos cuadrados baho diferentes perspectivas y como 
\vspace{0.5cm}



\vspace{21cm}




\section{Teor\'ia}
\vspace{1.5cm}


\subsection{¿Qu\'e es el M\'etodo de M\'inimos Cuadrados?}\vspace{0.5cm} 

Al enfrentarnos al siguiente problema $Ax=b$ donde $A\in\mathbb{R}^{mxn}$ y  conocemos el vector de datos $b$, el m\'etodo de min\'imos cuadrados ofrece una forma de minimizaci\'on del vector residual $r=b-Ax$ minimizando la norma al cuadrado del vector residual $\langle r,r \rangle$.

Esta forma de optimizar el error es atractiva por varias razones una de ellas es que a funci\'on de $||.||^2$ es una funci\'on suave.

Entonces es f\'acil ver que dados los datos este m\'etodo de una manera muy eficaz nos da la "mejor aproximaci\'on" posible para todos los puntos a la cual llamaremos modelo
\vspace{0.7cm}

%------------------------------------------------------------------------- 


\begin{center}
    \section{Perspectiva Geom\'etrica}
\end{center}
\vspace{0.7cm}
Una manera interesante de ver el m\'etodo de m\'nimos cuadrados es notar que la soluci\'on de este problema se deja ver como como la proyeccci\'on  de $b$ en el espacio columna de $A$.\\

Ahora cuando el sistema $Ax=b$(donde $b$ es el vector dado por los datos) no tiene una soluci\'on exacta es decir que $b\notin col(A)$ y buscamos su mejor aproximaci\'on
utilizamos el m\'etodo de m\'inimos, cuadrados donde
minimizaremos el vector residual $r=b-Ax$.\\

Ahora si tomamos el vector $\overline{b}\in col(A)$ como el vector m\'as cercano a $b$ en el subespacio $col(A)$  tendriamos que $\overline{b}$ es una buena aproxicimaci\'on que soluciona nuestro sistema, este vector $\overline{b}$ podemos mostrar que se deja ver como la proyecci\'on  de $b$ en el subespacio $col(A)$  tal que cumple lo siguiente 

\[
\min_{x \in \text{col}(A)} \| b - x \|
\]

Note que si $\overline{b}=Ax$ por lo tanto $r=b-\overline{b}$ luego por definicion de la proyecci\'on  se sigue que $r$ es ortogonal a $col(A)$ es decir que $A^\top r=0$ 

asumimos la ortogonalidad de $r$ pues si $r$ no fuera ortogonal a $\overline{b}$ esto significar\'ia que existe una mejor aproximaci\'on en $col(A)$ para $b$ por lo tanto no cumplir\'ia la condici\'on de minimizaci\'on y la proyecci\'on se deducir\'ia  de este hecho.

As\'i tendr\'iamos que $\overline{b}=Ax$, por definici\'on tendr\'iamos que $r=b-\overline{b}$es decir que $b=r+\overline{b}$, si aplicaramos la propiedad del triangulo esto se ver\'ia de la siguiente manera.


\vspace{0.5cm}


 
\vspace{6.5cm}
 



%-------------------------------------------------------------------------

\begin{center}    
\section{Perspectiva Algebraica}
\end{center}

 La ecuaci\'on normal del metodo de minimos cuadrados se obtiene de minimizar la suma de los vectores resuiduo al cuadrado.

 Supongamos que $Ax=\overline{b}$ define el modelo para nuestra regresi\'on lineal.

 Ahora nuestro objetivo es minimizar la norma vectores residuo al cuadradola cual llamaremos  $S(x)$ donde

 \begin{align*}
     S(x)=\langle r,r\rangle\\
     =(b-\overline{b})alaT(b-\overline{b})\\
     =(b-Ax)alaT(b-Ax)\\  
     =b^\top b - b^\top Ax - x^\top A^\top b + x^\top A^\top A x\\
     =b^\top b - 2 x^\top A^\top b + x^\top A^\top A x
 \end{align*}
Ahora minimicemos $S(x)$

\begin{align*}
    \frac{\partial S(x)}{\partial x} = -2A^\top b + 2A^\top Ax=0\\
    A^\top Ax = A^\top b
\end{align*}

Ahora suponiendo que $A^\top A$ es invertible tendr\'iamos que

\begin{equation}
    x =(A^\top A)^{-1} A^\top b
\end{equation}
 
A la ecuaci\'on $(1)$ la llamaremos la ecuaci\'on normal del m\'etodo de mi\'imos cuadrados

\begin{center}
\section{Conclusiones}\end{center}
\begin{itemize}
\item El m\'etodo de m\'inimos cuadrados es equivalente a proyectar el vector de los datos en el espacio columna de la matriz asociada a nuestro sistema lineal

\item La ecuaci\'on normal es la derivada de la suma de los vectores residuos

\item 
    
\end{itemize}

\section*{Bibliografía}


\end{document}

